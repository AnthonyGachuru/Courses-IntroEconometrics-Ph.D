%------------------------------------------------------My packages--------------------------------------------------

\usepackage{todonotes}                         %I use this to include comments on the PDF document. 
                                                                     %Requires todonotes.sty
                                                                     

\usepackage[longnamesfirst]{natbib}  % I use this package for the bibliography
\setlength{\bibsep}{6.5pt}  

\usepackage{setspace}                          %This package allows us to use singlespacing,doublespacing,
\usepackage{amsfonts}                          %The standard math fonts (American Math Society)
\usepackage{amsmath}                          %Mathematics package
\usepackage{amssymb}                         %Symbols (American Math Society)
\usepackage{amsthm}                           % Theorem enviroment (American Math society)
\usepackage{bm}                                    %Bold greeks
\usepackage{bbm}                                  %Bold numerals
\usepackage{color}                                 %Good for introducing texts of different colors
	\definecolor{MyDarkBlue}{rgb}{0.1,0.2,0.65}
\usepackage{hyperref}                          %Controls hyperreferences
	\hypersetup{linkcolor=black, citecolor=MyDarkBlue, urlcolor=black, colorlinks=true}
\usepackage{mathtools}                        % I copied this from Tomasz (not sure what it does)
\usepackage{graphicx}                          % Idem

\usepackage{subfig}                               %This is the package I use for inserting multiple figures
\captionsetup[subfloat] {position=bottom} 

\usepackage{enumerate}                      %Item evironments
\usepackage[vmargin={1.5in, 1.5in},hmargin={1in, 1in}]{geometry} 


\usepackage[T1]{fontenc}

\usepackage{lmodern} \normalfont %to load T1lmr.fd 
\DeclareFontShape{T1}{lmr}{bx}{sc} { <-> ssub * cmr/bx/sc }{}

%\usepackage[labelfont=bf, margin=1cm]{caption}

\usepackage{pst-func}

\numberwithin{equation}{section}



%--------------------------------------------------New Theorems---------------------------------------------------

\newtheorem{lemma}{Lemma}
\newtheorem*{lemma*}{Lemma}
\newtheorem{cor}{Corollary}
\newtheorem{theorem}{Theorem}
\newtheorem{proposition}{Proposition}
\newtheorem{conjecture}{Conjecture}
\newtheorem*{theorem*}{Theorem}
\newtheorem{result}{Result}
\newtheorem{claim}{Claim}
\newtheorem*{coro}{Corollary}
\theoremstyle{definition}
\newtheorem{problem}{Problem}
\newtheorem{remark}{Remark}
\newtheorem{axiom}{Axiom}
\newtheorem*{definition}{Definition}
\newtheorem{assumption}{Assumption}
\newtheorem*{axiom*}{Axiom}
\newtheorem*{comment*} {Comment}
\newtheorem{example}{Example}

%---------------------------------------------Some commands for Autoreference------------------------------

\newcommand{\refaxiom}[1]{\hyperref[{#1}]{Axiom }\autoref{#1}}
\newcommand{\refass}[1]{\hyperref[{#1}]{Assumption }\autoref{#1}}
\newcommand{\refres}[1]{\hyperref[{#1}]{Result }\autoref{#1}}
\newcommand{\refprop}[1]{\hyperref[{#1}]{Proposition }\autoref{#1}}
\newcommand{\refcor}[1]{\hyperref[{#1}]{Corollary }\autoref{#1}}
\newcommand{\reflemma}[1]{\hyperref[{#1}]{Lemma }\autoref{#1}}
\newcommand{\refremark}[1]{\hyperref[{#1}]{Remark }\autoref{#1}}
\newcommand{\reftheorem}[1]{\hyperref[{#1}]{Theorem}\autoref{#1}}



  
%--------------------------------------------Shortcuts for math symbols--------------------------------------------

\newcommand{\F}{\mathcal{F}}                                               %Sigma algebra
\newcommand{\T}{\mathcal{T}}                                               %Topology  
\newcommand{\R}{\mathbb{R}}                                              %Real Numbers
\newcommand{\cdist}{\overset{d}{\rightarrow}}                    %Convergence in distributions
\newcommand{\cprob}{\overset{p}{\rightarrow}}                  %Convergence in probability
\newcommand{\cas}{\overset{a.s}{\rightarrow}}                   %Almost sure convergence
\newcommand{\prob}{\mathbb{P}}                                         %Bold probability
\newcommand{\expec}{\mathbb{E}}                                      %Bold Expectation
\newcommand{\e}{\varepsilon}                                                %epsilon


\newcommand\cites[1]{\citeauthor{#1}'s \citeyearpar{#1}}               %Possesive cite (I never use it)

%--------------------------------------------Commands I have used before-----------------------------------
\newcommand{\y}{y}                                       % Outcome Variable
\newcommand{\Y}{X}                                      % Endogenous regressor
\newcommand{\Ybar}{\textbf{Y}}               % Matrix of Endogenours Regressors
\newcommand{\vbar}{\textbf{V}}               % Matrix of Reduced Form Errors
\newcommand{\Gammab}{\mathbf{\Gamma}}               % Unrestricted coefficients
\newcommand{\w}[1]{\ddot{#1}}                    % Within Transformation
\newcommand{\0}{\textbf{0}}                          % Bold 0
\newcommand{\eye}[1]{\mathbb{I}_{#1}}      %Identity matrix
\newcommand{\N}{\mathcal{N}}                     % Cal N (for Normal)
\newcommand{\kron}{\otimes}                        %Kroenecker
\newcommand{\norm}{\rho}                             %Rho parameter in Gary's re-parameterization
\newcommand{\vect}{\text{vec}}                       %Text vectorization 
\newcommand{\SX}{\mathbf{X}}
\newcommand{\Comment}[1]{\todo[inline, color=green!40]{\textbf{Comment: }#1}}         %Creates a green box for my comments
\newcommand{\Remark}[1]{\todo[inline, color=blue!40]{\textbf{Remark: }#1}}         %Creates a green box for my comments
\newcommand{\kronmean}[2]{\left(\begin{array}{c} #1 \\1 \end{array}\right) \kron #2 }                            %kronecker mean (beta 1) kron something
\newcommand{\p}{\bot}                                     %the projection command
\newcommand{\kronmeanzero}[2]{\left(\begin{array}{c} #1 \\0 \end{array}\right) \kron #2 }                            %kronecker mean (beta 0) kron something


%----------------------------------------Commands for this paper---------------------------------------
\newcommand{\x}{x}                                                           %Realization
\newcommand{\X}{X}                                                         %Random Variable
\newcommand{\SW}{\mathcal{W}}                                     %Sample Space
\newcommand{\ParamS}{\mathbf{\Theta}}                    %Parameter Space

\newcommand{\W}{\textbf{W}}                                          %Weighting Function

\newcommand{\Borel}[1]{\mathcal{B}({#1})}                    %Borel sigma algebra
 
%A double line
\def\doubleline{
\begin{center}
\line(1,0){400}\\
\line(1,0){400}
\end{center}
}

%An "Important" header
\def\Important{
\begin{center}
\line(1,0){480}\\
\textcolor{red}{Important}
\line(1,0){480}
\end{center}
}



\newcommand{\Ideas}[1]{\todo[inline]{\textbf{Ideas: }#1}}         %Creates an orange box fot the main ideas of each section. 
