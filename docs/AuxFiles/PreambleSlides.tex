%The pst-fun package is very convenient for plotting mathematical functions
%The package ragged2e is used to "justify" the text in the slides
\usepackage{beamerthemetree}
 \usepackage{graphics}
\usepackage{pst-func}
\usepackage{ragged2e} 
\usepackage{centernot}
%My standard preamble
\usepackage[longnamesfirst]{natbib}  
\renewcommand{\bibsection}{\subsubsection*{\bibname } }
\setlength{\bibsep}{6.5pt}
\newcommand\cites[1]{\citeauthor{#1}'s \citeyearpar{#1}}
\usepackage{setspace}
\usepackage{amsfonts}
\usepackage{amssymb}
\usepackage{color}
	\definecolor{MyDarkBlue}{rgb}{0,0.38,0.85}
\usepackage{hyperref}
	\hypersetup{linkcolor=MyDarkBlue, citecolor=MyDarkBlue, colorlinks=true}
\usepackage{mathtools}
\usepackage{graphicx}
\usepackage{amsthm}
\usepackage{amsmath}
\usepackage{enumerate}
%Pre-amble for the figures
\usepackage[format=plain,labelfont=bf,up,textfont=it,up]{caption}
\usepackage{subfigure}                               %This is the package I use for inserting multiple figures
\captionsetup[subfloat] {position=bottom} 
\usepackage{fancybox}
%And some theorem environments
\newtheorem{cor}{Corollary}
\newtheorem*{theorem*}{Theorem}
\newtheorem{result}{Result}
\newtheorem{claim}{Claim}
\newtheorem*{coro}{Corollary}
\theoremstyle{definition}
\newtheorem{remark}{Remark}
\newtheorem{axiom}{Axiom}
\newtheorem{assumption}{Assumption}
\newtheorem{proposition}{Proposition}
\newtheorem*{axiom*}{Axiom}
\newcommand{\refaxiom}[1]{\hyperref[{#1}]{Axiom }\autoref{#1}}
\newcommand{\refass}[1]{\hyperref[{#1}]{Assumption }\autoref{#1}}
\newcommand{\refres}[1]{\hyperref[{#1}]{Result }\autoref{#1}}
\newcommand{\reflemma}[1]{\hyperref[{#1}]{Lemma }\autoref{#1}}
\newcommand{\refprop}[1]{\hyperref[{#1}]{Proposition }\autoref{#1}}
\newcommand{\refcor}[1]{\hyperref[{#1}]{Corollary }\autoref{#1}}
\newcommand{\refdef}[1]{\hyperref[{#1}]{Definition }\autoref{#1}}

%AdditionalStuff and Shortcuts
\usetheme{Singapore}
\setbeamertemplate{footline}[frame number]
%\useoutertheme{progressbar}
\usecolortheme{seagull}

\usepackage{pgfpages}
%\setbeameroption{show notes}
%\setbeameroption{show notes on second screen}

%These are some of the commands that I use all the time
\usepackage{manfnt}
%--------------------------------------------Shortcuts for math symbols--------------------------------------------
\newcommand{\F}{\mathcal{F}}                                            %Sigma algebra
\newcommand{\T}{\mathcal{T}}                                            %Topology  
\newcommand{\R}{\mathbb{R}}                                            %Real Numbers
\newcommand{\cdist}{\overset{d}{\rightarrow}}                     %Convergence in distributions
\newcommand{\cprob}{\overset{p}{\rightarrow}}                   %Convergence in probability
\newcommand{\cas}{\overset{a.s}{\rightarrow}}                    %Almost sure convergence
\newcommand{\prob}{\mathbb{P}}                                         %Bold probability
\newcommand{\expec}{\mathbb{E}}                                      %Bold Expectation
\newcommand{\y}{y}                                                              % Outcome Variable
\newcommand{\X}{X}                                                             % Exogenous regressor    
\newcommand{\Y}{Y}                                                             % Endogenous regressor
\newcommand{\w}[1]{#1}                                                       % Within Transformation
\newcommand{\kron}{\otimes}                                               %Kroenecker
\newcommand{\eye}[1]{\mathbb{I}_{#1}}      			   %Identity matrix
\newcommand{\caps}[1]{{\scshape{#1}}}      			   %Identity matrix
\newcommand{\p}{\bot}                                    			   %Projection matrix
\newcommand{\kronmean}[2]{\left(\begin{array}{c} #1 \\1 \end{array}\right) \kron #2 }         
%kronecker mean (beta 1) kron something