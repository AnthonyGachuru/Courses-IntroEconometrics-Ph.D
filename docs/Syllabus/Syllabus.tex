\documentclass[11pt]{article} 
\usepackage{etex}
%------------------------------------------------------My packages--------------------------------------------------

\usepackage{todonotes}                         %I use this to include comments on the PDF document. 
                                                                     %Requires todonotes.sty
                                                                     

\usepackage[longnamesfirst]{natbib}  % I use this package for the bibliography
\setlength{\bibsep}{6.5pt}  

\usepackage{setspace}                          %This package allows us to use singlespacing,doublespacing,
\usepackage{amsfonts}                          %The standard math fonts (American Math Society)
\usepackage{amsmath}                          %Mathematics package
\usepackage{amssymb}                         %Symbols (American Math Society)
\usepackage{amsthm}                           % Theorem enviroment (American Math society)
\usepackage{bm}                                    %Bold greeks
\usepackage{bbm}                                  %Bold numerals
\usepackage{color}                                 %Good for introducing texts of different colors
	\definecolor{MyDarkBlue}{rgb}{0.1,0.2,0.65}
\usepackage{hyperref}                          %Controls hyperreferences
	\hypersetup{linkcolor=black, citecolor=MyDarkBlue, urlcolor=black, colorlinks=true}
\usepackage{mathtools}                        % I copied this from Tomasz (not sure what it does)
\usepackage{graphicx}                          % Idem

\usepackage{subfig}                               %This is the package I use for inserting multiple figures
\captionsetup[subfloat] {position=bottom} 

\usepackage{enumerate}                      %Item evironments
\usepackage[vmargin={1.5in, 1.5in},hmargin={1in, 1in}]{geometry} 


\usepackage[T1]{fontenc}

\usepackage{lmodern} \normalfont %to load T1lmr.fd 
\DeclareFontShape{T1}{lmr}{bx}{sc} { <-> ssub * cmr/bx/sc }{}

%\usepackage[labelfont=bf, margin=1cm]{caption}

\usepackage{pst-func}

\numberwithin{equation}{section}



%--------------------------------------------------New Theorems---------------------------------------------------

\newtheorem{lemma}{Lemma}
\newtheorem*{lemma*}{Lemma}
\newtheorem{cor}{Corollary}
\newtheorem{theorem}{Theorem}
\newtheorem{proposition}{Proposition}
\newtheorem{conjecture}{Conjecture}
\newtheorem*{theorem*}{Theorem}
\newtheorem{result}{Result}
\newtheorem{claim}{Claim}
\newtheorem*{coro}{Corollary}
\theoremstyle{definition} % -------------------
\newtheorem{problem}{Problem}
\newtheorem{remark}{Remark}
\newtheorem{axiom}{Axiom}
\newtheorem*{definition}{Definition}
\newtheorem{assumption}{Assumption}
\newtheorem*{axiom*}{Axiom}
\newtheorem*{comment*} {Comment}
\newtheorem{example}{Example}
\newtheorem{prproblem}{Practice Problem}

%---------------------------------------------Some commands for Autoreference------------------------------

\newcommand{\refaxiom}[1]{\hyperref[{#1}]{Axiom }\autoref{#1}}
\newcommand{\refass}[1]{\hyperref[{#1}]{Assumption }\autoref{#1}}
\newcommand{\refres}[1]{\hyperref[{#1}]{Result }\autoref{#1}}
\newcommand{\refprop}[1]{\hyperref[{#1}]{Proposition }\autoref{#1}}
\newcommand{\refcor}[1]{\hyperref[{#1}]{Corollary }\autoref{#1}}
\newcommand{\reflemma}[1]{\hyperref[{#1}]{Lemma }\autoref{#1}}
\newcommand{\refremark}[1]{\hyperref[{#1}]{Remark }\autoref{#1}}
\newcommand{\reftheorem}[1]{\hyperref[{#1}]{Theorem}\autoref{#1}}



  
%--------------------------------------------Shortcuts for math symbols--------------------------------------------

\newcommand{\F}{\mathcal{F}}                                               %Sigma algebra
\newcommand{\T}{\mathcal{T}}                                               %Topology  
\newcommand{\R}{\mathbb{R}}                                              %Real Numbers
\newcommand{\cdist}{\overset{d}{\rightarrow}}                    %Convergence in distributions
\newcommand{\cprob}{\overset{p}{\rightarrow}}                  %Convergence in probability
\newcommand{\cas}{\overset{a.s}{\rightarrow}}                   %Almost sure convergence
\newcommand{\prob}{\mathbb{P}}                                         %Bold probability
\newcommand{\expec}{\mathbb{E}}                                      %Bold Expectation
\newcommand{\e}{\varepsilon}                                                %epsilon


\newcommand\cites[1]{\citeauthor{#1}'s \citeyearpar{#1}}               %Possesive cite (I never use it)

%--------------------------------------------Commands I have used before-----------------------------------
\newcommand{\y}{y}                                       % Outcome Variable
\newcommand{\Y}{X}                                      % Endogenous regressor
\newcommand{\Ybar}{\textbf{Y}}               % Matrix of Endogenours Regressors
\newcommand{\vbar}{\textbf{V}}               % Matrix of Reduced Form Errors
\newcommand{\Gammab}{\mathbf{\Gamma}}               % Unrestricted coefficients
\newcommand{\w}[1]{\ddot{#1}}                    % Within Transformation
\newcommand{\0}{\textbf{0}}                          % Bold 0
\newcommand{\eye}[1]{\mathbb{I}_{#1}}      %Identity matrix
\newcommand{\N}{\mathcal{N}}                     % Cal N (for Normal)
\newcommand{\kron}{\otimes}                        %Kroenecker
\newcommand{\norm}{\rho}                             %Rho parameter in Gary's re-parameterization
\newcommand{\vect}{\text{vec}}                       %Text vectorization 
\newcommand{\SX}{\mathbf{X}}
\newcommand{\Comment}[1]{\todo[inline, color=green!40]{\textbf{Comment: }#1}}         %Creates a green box for my comments
\newcommand{\Remark}[1]{\todo[inline, color=blue!40]{\textbf{Remark: }#1}}         %Creates a green box for my comments
\newcommand{\kronmean}[2]{\left(\begin{array}{c} #1 \\1 \end{array}\right) \kron #2 }                            %kronecker mean (beta 1) kron something
\newcommand{\p}{\bot}                                     %the projection command
\newcommand{\kronmeanzero}[2]{\left(\begin{array}{c} #1 \\0 \end{array}\right) \kron #2 }                            %kronecker mean (beta 0) kron something


%----------------------------------------Commands for this paper---------------------------------------
\newcommand{\x}{x}                                                           %Realization
\newcommand{\X}{X}                                                         %Random Variable
\newcommand{\SW}{\mathcal{W}}                                     %Sample Space
\newcommand{\ParamS}{\mathbf{\Theta}}                    %Parameter Space

\newcommand{\W}{\textbf{W}}                                          %Weighting Function

\newcommand{\Borel}[1]{\mathcal{B}({#1})}                    %Borel sigma algebra
 
%A double line
\def\doubleline{
\begin{center}
\line(1,0){400}\\
\line(1,0){400}
\end{center}
}

%An "Important" header
\def\Important{
\begin{center}
\line(1,0){480}\\
\textcolor{red}{Important}
\line(1,0){480}
\end{center}
}



\newcommand{\Ideas}[1]{\todo[inline]{\textbf{Ideas: }#1}}         %Creates an orange box fot the main ideas of each section. 


\title {Econometrics I\thanks{First version: August 26th, 2013. This version: \today} }
\author {Jos\'e Luis Montiel Olea \protect\\
Fall 2019
}
\date{}
             
\begin{document}
\onehalfspace
\maketitle


\noindent \textbf{Course Description:} This course (September 4th-October 15th) will provide a graduate-level introduction to probability and statistics for economists.\\ 

\noindent \textbf{Prerequisite(s):} My main assumption is that you have been exposed to at least one proof-based course in economics or mathematics. Previous undergraduate work in probability and statistics is neither assumed nor required. \\ 

\noindent \textbf{Background:} Familiarity with the following concepts would be helpful: sequences of real numbers, convergence of a sequence of real numbers, euclidean spaces,  continuity of $\R^m$-valued functions, integration. \\

\noindent \textbf{Text(s):} The lectures will be based on a set of notes and slides posted online. Ocassionally, the following references will be cited:

\begin{enumerate}[a)]
\item ``\emph{Probability: Theory and Examples}'' by Rick Durret [2010]. Cambridge University Press. Fourth Edition
\item ``\emph{Testing Statistical Hypothesis}'' by Lehmann, E.L. and Romano, J.P. [2005]. Springer Verlag
\item ``\emph{Probability Theory with Economic Applications}'' by Efe Ok. \href{https://files.nyu.edu/eo1/public/books.html}{Available online}.
\item ``\emph{Introduction to Mathematical Statistics}'' by Hogg, Robert V. and Mckean, Joseph W. and Allen, Craig D. [2006] Pearson Education India. 
\item ``\emph{Real analysis and Probability}'' by Dudley, Richard [2002]. Cambridge University Press. 
\item ``\emph{Probability and Measure}'' by Billingsley, Patrick [1995]. John Wiley \& Sons. 
\item ``\emph{Introduction to Mathematical Statistics: a Decision Theoretic Approach}'' by Ferguson, Thomas [1967]. Academic Press New York
\end{enumerate}
\noindent These are optional references that you can find helpful during the course.  \\

\noindent \textbf{Evaluation:} There will be 5 problem sets. Each of them will have a weight of $12\%$ in the final grade. The remaining $40\%$ will be based on a final exam. This means that the problem sets are important. \\ 

\noindent \textbf{Lecture Notes and Slides:} The lecture notes and slides were prepared with two different styles in mind. The slides were prepared to provide an overview of the lectures; thus, you should not expect to learn specific details from them. The lecture notes were intended to provide a thorough explanation of the materials covered in class. I expect you to look at the notes carefully, using the slides to get a sense of what the goal of the lecture is. 

\newpage

\textbf{Course Outline (Tentative)}

\begin{enumerate}
\item {\scshape Probability Theory} (4 lectures/2 weeks)
\begin{itemize}
\item \emph{Lecture 1-2}: Probability Spaces, Real-valued Random Variables, Cumulative Distribution Functions (c.d.f.),  Probability Density Functions (p.d.f.), Moments of Random Variables, Mean and Variance, Moment Generating Function. \\

{\scshape Appendix:} If I have time, I will give have a very quick discussion concerning \emph{distance between probability measures}. \\

\item \emph{Lecture 3-4} Multivariate Distributions, Moments of Random Vectors, Independence of Random Variables and some useful characterizations, Conditional Probability and Conditional Expectation.  \\

{\scshape Appendix:} Kolmogorov\textquoteright s definition of Independence. \\
\end{itemize}
\item {\scshape Mathematical Statistics} (8 lectures/4 weeks) (This is my favorite part!)
\begin{itemize}
\item \emph{Lecture 5-6:} Definition of a Statistical Model (I will also talk about identification, statistical sufficiency, and I will introduce the Normal Linear Regression Model as a running example). Definition of a Statistical Problem (general framework to talk about estimation, hypothesis testing, and inference). Elements of a finite-sample statistical decision problem: loss function, Risk, and decision rules. Bayes rules and minimax rules.   \\

\item \emph{Lecture 7-8:} Point estimation. Definition of Maximum Likelihood (ML) Estimators and Bayesian Estimators.  Derivation of ML and Bayes estimators for the Normal Linear Regression model (Ordinary Least Squares and Ridge).  Mean squared error optimality of OLS among unbiased estimators. Cramer-Rao lower bound. Comments regarding some suboptimality results for the OLS estimator (James-Stein/Empirical Bayes estimators). \\

\item \emph{Lecture 9-10:} Testing. Null hypothesis and alternative hypothesis. Score Test, Likelihood Ratio Test, Wald tests, and Bayes Tests. Finite-sample properties of statistical tests (rate of Type I/Type II error). Neyman-Pearson Lemma and the optimality of the Likelihood Ratio Test. Testing Problems with a nuisance parameter. \\

\item \emph{Lecture 11-12:} Confidence sets and credible sets. (Parametric) Bootstrap confidence intervals and Bayesian equal-tailed credible intervals. Finite-sample properties of confidence/credible sets. Constructing confidence sets via test-inversion. Confidence bands. 

\end{itemize}

\end{enumerate}

\bibliographystyle{../AuxFiles/ecta}
\bibliography{../AuxFiles/BibMaster}

\end{document}





 
 