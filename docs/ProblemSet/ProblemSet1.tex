%%!TEX TS-program = latex
\documentclass[11pt]{article} %DIF > 
\usepackage{etex}
\usepackage[utf8]{inputenc}
\input ../AuxFiles/PreambleNotes.tex

\begin{document}
\onehalfspace

\textbf{Problem Set 1 (Lectures 1 and 2)}

\begin{enumerate}
\item (The $\sigma$-algebra generated by a collection of sets, \textbf{Optional Problem}) Let $\Omega$ be an arbitrary non-empty set and let $A$ be a collection of elements of $2^\Omega$.  Define 

$$F^*(A) \equiv \{\mathcal{F} \: | \: \mathcal{F} \: \text{is a} \: \sigma\text{-algebra of } \Omega \: \text{containing} \: A\} $$

\begin{enumerate}[i)]
\item Show that $F^*(A)$ is non-empty. 
\item Let $\sigma(A)$ denote the intersection over all the $\sigma$-algebras contained in $F^*(A)$. Show that $\sigma(A)$ is a $\sigma$-algebra.
\end{enumerate}

{\scshape Comment:} Skip this if it does not sound interesting to you. 

\noindent {\scshape Optional:} It turns out that there is no other $\sigma$-algebra  $\mathcal{F}$ such that $A \subseteq \mathcal{F}$ and $\mathcal{F} \subset \sigma(A)$ (you can also show this if you are interested). The set $\sigma(A)$ is the (unique) smallest $\sigma$- algebra containing $A$.\\

\item  (\textbf{Optional Problem}) Show that $\sigma$-algebra is ``closed'' under countable intersections. That is, if $F_n \in \mathcal{F}$ for all $n\in\mathbb{N}$ then $\cap_{n=1}^\infty F_n \in \mathcal{F}$.



\item (Properties of a probability measure) Let $(\Omega, \mathcal{F}, \prob)$ be a probability space. Show that:
\begin{enumerate}
\item $\prob(F_1 \cup F_2) = \prob(F_1) + \prob(F_2) - \prob(F_1 \cap F_2)$ \text{for any } $F_1,F_2 \in \mathcal{F}$
\item $\prob(\cup_{n \in \mathbb{N}} F_n) \leq \sum_{n \in \mathbb{N}} \prob(F_n) $ for any countable collection $\{F_n\}$
\end{enumerate}

\noindent {\scshape Comment:} These are useful properties implied by the definition of probability measure. We will use some of them throughout the course. 

\item Proof the following Proposition: If $F_{X}$ is the c.d.f. of a random variable $X: \Omega \rightarrow \R$ then 
\begin{enumerate}
\item $F_{X}$ is non-decreasing.
\item $\lim_{x \uparrow \infty} F_{X}(x)=1$
\item $\lim_{x \downarrow -\infty} F_{X}(x)=0$
\item $\lim_{h \rightarrow 0^{+}} F(x+h)=F(x)$
\end{enumerate}

\noindent {\scshape Comment:} Combined with the optional part below, this gives a full characterization of how c.d.f.s for real-valued random variables can look.

\noindent ({\scshape Optional}) Furthermore, if $F$ is a function satisfying 1,2,3,4, then there is a probability space $(\Omega, \mathcal{F}, \prob)$ and a random variable $X: \Omega \rightarrow \R$ such that $F$ coincides with $F_{X}$. 

\item (Moments of some common distributons) Use the definitions of expectations provided in class to solve the following problems:
\begin{enumerate}
\item Show that if $X$ is a Bernoulli random variable with parameter $p$, then $\expec_{F}[X]=p$ and $\expec_{F}[(X-p)^2]=p(1-p)$
\item Show that if $X$ is a Normal Random variable with parameters $\mu$ and $\sigma^2$ then $\expec_{F}[X]=\mu$ and $\expec_{F}[(X-\mu)^2]=\sigma^2$.\footnote{Here you can use the fact that $\int_{-\infty}^{\infty} u (1/\sqrt{2 \pi}) e^{-u^2/2} du =0$ and $\int_{-\infty}^{\infty} u^2 (1/\sqrt{2 \pi}) e^{-u^2/2} du =1$.} 
\item Show that if $X$ is a Pareto distribution with parameters $(x_m, \alpha)$, then for any $n \geq \alpha,$ $\expec_{F}[X^{n}]=\infty$. 
\item Show that the moment generating function of a Normal random variable with parameters $\mu$ and $\sigma^2$ is given by:
$$\mu_{X}(t)= \exp \Big( t \mu + \frac{t^2 \sigma^2}{2}  \Big)  $$

\item Show that if $X \sim \mathcal{N}(0,1)$, then the random variable $Y:\R \rightarrow \R$ given by $\mu + \sigma X$ has the c.d.f. of Normal random variable with parameters $(\mu, \sigma^2)$. 

\end{enumerate}

\item (95\% of the mass within 1.96 standard deviations)Let $X \sim \mathcal{N}(0,1)$. Convince yourself that 
$$\prob_{X}(|X|<1.96) = \int_{-1.96}^{1.96} \frac{1}{\sqrt{2\pi}} \exp\Big( -\frac{1}{2}x^2 \Big)dx $$
\noindent and convince yourself that it is hard to give an analytic solution for this integral. 
\begin{enumerate}[a]
\item Go to matlab and use the command \texttt{randn(10000,1)} to get a vector with $10,000$ realizations of a standard normal random variable.   Report the share of the realizations that fall in the interval $[-1.96,1.96]$. 
\item Now, suppose that $X \sim \mathcal{N}(\mu,\sigma^2)$. Use the result in a) to give an approximation for the value of 
$$ \prob_{X}(|X-\mu|< 1.96 \sigma) $$ 
\end{enumerate}

\end{enumerate}

\newpage

%\bibliographystyle{../AuxFiles/ecta}
%\bibliography{../AuxFiles/BibMaster}

\end{document}